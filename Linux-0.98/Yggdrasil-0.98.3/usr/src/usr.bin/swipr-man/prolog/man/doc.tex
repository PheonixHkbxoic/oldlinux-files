\documentstyle[manual,twoside,psfig]{report}

\newcommand{\predicate}[3]{\item[#1({\it #3})]\hfil\mbox{}\\}
\newcommand{\prefix}[2]{\item[#1 {\it #2}]\hfil\mbox{}\\}
\newcommand{\infix}[3]{\item[{\it #1} #2 {\it #3}]\hfil\mbox{}\\}
\newcommand{\noargpredicate}[1]{\item[#1]\hfil\mbox{}\\}
\newcommand{\function}[4]{\item[{\it #1}] {\bf #2(\it #4})\\}
\newcommand{\noargfunction}[2]{\item[{\it #1}] {\bf #2()}\\}
\newcommand{\macro}[4]{\item[{\it #1}] {\bf #2(\it #4})\\}
\newcommand{\noargmacro}[2]{\item[{\it #1}] {\bf #2}\\}
\newcommand{\variable}[2]{\item[{\it #1}] {\bf #2}\\}
\newcommand{\tty}[1]{\mbox{\verb@#1@}}
\newcommand{\bug}[1]{\footnote{BUG: #1}}
\newcommand{\version}{1.5.0, August 1990}
\newcommand{\pow}[2]{{#1}^{#2}}
\itemsep 0pt

\makeindex
\sloppy
%\psdraft

%\includeonly{intro}

\begin{document}

\begin{titlepage}

\newlength{\uvawidth}
\settowidth{\uvawidth}{\LARGE University of Amsterdam}

\newcommand{\uvaaddress}{%
\parbox[b]{\uvawidth}{%
    \begin{center}
	\LARGE
	      University of Amsterdam \\[3mm]
	\small
	Dept. of Social Science Informatics (SWI) \\
	    Herengracht 196, 1016 BS~~ Amsterdam \\
		    The Netherlands \\
		Tel. (+31) 20 5252073
    \end{center}}}

\mbox{}\vspace{-1in}
\parbox{\textwidth}{%
    \makebox[\textwidth]{%
	\UvA{1in}%
	\hfill%
	\raisebox{-12pt}{\uvaaddress}
	\hfill%
	\SWI{1in}}
}
\vfil\vfil\vfil
\begin{center}
	{\Huge \bf SWI-Prolog 1.5	\\[3mm]
	 \LARGE Reference Manual}	\\[1.5cm]
	{\large \it Jan Wielemaker}	\\[7mm]
	{\large jan@swi.psy.uva.nl}
\end{center}
\vfil
\begin{quote}
SWI-Prolog is a Prolog implementation based on a subset of the WAM
(Warren Abstract Machine, \cite{Warren:83b}). SWI-Prolog has been
designed and implemented such that it can easily be modified for
experiments with logic programming and the relation between logic
programming and other programming paradigms (such as the object
oriented PCE environment, \cite{P1098:C1.6}).  SWI-Prolog has a rich
set of built-in predicates and reasonable performance, which makes it
possible to develop substantial applications in it.  The current
version offers a module system, garbage collection and an interface to
the C language.

This document gives an overview of the features, system limits and
built-in predicates.
\end{quote}
\vfil
\vfil
\begin{quote}
Copyright \copyright\ 1990 Jan Wielemaker
\end{quote}
\end{titlepage}

\pagestyle{esprit}
\newcommand{\bottomleft}{\mbox{}}
\newcommand{\bottomright}{\mbox{}}

{\parskip 0pt
\tableofcontents
}

\include{intro}
This file is builtin.def, from which is created builtin.c.
It implements the builtin "builtin" in Bash.

Copyright (C) 1987, 1989, 1991 Free Software Foundation, Inc.

This file is part of GNU Bash, the Bourne Again SHell.

Bash is free software; you can redistribute it and/or modify it under
the terms of the GNU General Public License as published by the Free
Software Foundation; either version 1, or (at your option) any later
version.

Bash is distributed in the hope that it will be useful, but WITHOUT ANY
WARRANTY; without even the implied warranty of MERCHANTABILITY or
FITNESS FOR A PARTICULAR PURPOSE.  See the GNU General Public License
for more details.

You should have received a copy of the GNU General Public License along
with Bash; see the file COPYING.  If not, write to the Free Software
Foundation, 675 Mass Ave, Cambridge, MA 02139, USA.

$PRODUCES builtin.c

$BUILTIN builtin
$FUNCTION builtin_builtin
$SHORT_DOC builtin [shell-builtin [arg ...]]
Run a shell builtin.  This is useful when you wish to rename a
shell builtin to be a function, but need the functionality of the
builtin within the function itself.
$END

#include "../shell.h"
extern char *this_command_name;

/* Run the command mentioned in list directly, without going through the
   normal alias/function/builtin/filename lookup process. */
builtin_builtin (list)
     WORD_LIST *list;
{
  extern Function *find_shell_builtin (), *builtin_address ();
  Function *function;
  register char *command;

  if (!list)
    return (EXECUTION_SUCCESS);

  command = (list->word->word);
#if defined (DISABLED_BUILTINS)
  function = builtin_address (command);
#else /* !DISABLED_BUILTINS */
  function = find_shell_builtin (command);
#endif /* !DISABLED_BUILTINS */

  if (!function)
    {
      builtin_error ("%s: Not a shell builtin!", command);
      return (EXECUTION_FAILURE);
    }
  else
    {
      this_command_name = command;
      list = list->next;
      return ((*function) (list));
    }
}


\include{module}
\include{foreign}
\include{hack}
\include{summary}

\bibliographystyle{name}
\bibliography{abc,esprit,manual}

\documentstyle[manual,twoside,psfig]{report}

\newcommand{\predicate}[3]{\item[#1({\it #3})]\hfil\mbox{}\\}
\newcommand{\prefix}[2]{\item[#1 {\it #2}]\hfil\mbox{}\\}
\newcommand{\infix}[3]{\item[{\it #1} #2 {\it #3}]\hfil\mbox{}\\}
\newcommand{\noargpredicate}[1]{\item[#1]\hfil\mbox{}\\}
\newcommand{\function}[4]{\item[{\it #1}] {\bf #2(\it #4})\\}
\newcommand{\noargfunction}[2]{\item[{\it #1}] {\bf #2()}\\}
\newcommand{\macro}[4]{\item[{\it #1}] {\bf #2(\it #4})\\}
\newcommand{\noargmacro}[2]{\item[{\it #1}] {\bf #2}\\}
\newcommand{\variable}[2]{\item[{\it #1}] {\bf #2}\\}
\newcommand{\tty}[1]{\mbox{\verb@#1@}}
\newcommand{\bug}[1]{\footnote{BUG: #1}}
\newcommand{\version}{1.5.0, August 1990}
\newcommand{\pow}[2]{{#1}^{#2}}
\itemsep 0pt

\makeindex
\sloppy
%\psdraft

%\includeonly{intro}

\begin{document}

\begin{titlepage}

\newlength{\uvawidth}
\settowidth{\uvawidth}{\LARGE University of Amsterdam}

\newcommand{\uvaaddress}{%
\parbox[b]{\uvawidth}{%
    \begin{center}
	\LARGE
	      University of Amsterdam \\[3mm]
	\small
	Dept. of Social Science Informatics (SWI) \\
	    Herengracht 196, 1016 BS~~ Amsterdam \\
		    The Netherlands \\
		Tel. (+31) 20 5252073
    \end{center}}}

\mbox{}\vspace{-1in}
\parbox{\textwidth}{%
    \makebox[\textwidth]{%
	\UvA{1in}%
	\hfill%
	\raisebox{-12pt}{\uvaaddress}
	\hfill%
	\SWI{1in}}
}
\vfil\vfil\vfil
\begin{center}
	{\Huge \bf SWI-Prolog 1.5	\\[3mm]
	 \LARGE Reference Manual}	\\[1.5cm]
	{\large \it Jan Wielemaker}	\\[7mm]
	{\large jan@swi.psy.uva.nl}
\end{center}
\vfil
\begin{quote}
SWI-Prolog is a Prolog implementation based on a subset of the WAM
(Warren Abstract Machine, \cite{Warren:83b}). SWI-Prolog has been
designed and implemented such that it can easily be modified for
experiments with logic programming and the relation between logic
programming and other programming paradigms (such as the object
oriented PCE environment, \cite{P1098:C1.6}).  SWI-Prolog has a rich
set of built-in predicates and reasonable performance, which makes it
possible to develop substantial applications in it.  The current
version offers a module system, garbage collection and an interface to
the C language.

This document gives an overview of the features, system limits and
built-in predicates.
\end{quote}
\vfil
\vfil
\begin{quote}
Copyright \copyright\ 1990 Jan Wielemaker
\end{quote}
\end{titlepage}

\pagestyle{esprit}
\newcommand{\bottomleft}{\mbox{}}
\newcommand{\bottomright}{\mbox{}}

{\parskip 0pt
\tableofcontents
}

\include{intro}
This file is builtin.def, from which is created builtin.c.
It implements the builtin "builtin" in Bash.

Copyright (C) 1987, 1989, 1991 Free Software Foundation, Inc.

This file is part of GNU Bash, the Bourne Again SHell.

Bash is free software; you can redistribute it and/or modify it under
the terms of the GNU General Public License as published by the Free
Software Foundation; either version 1, or (at your option) any later
version.

Bash is distributed in the hope that it will be useful, but WITHOUT ANY
WARRANTY; without even the implied warranty of MERCHANTABILITY or
FITNESS FOR A PARTICULAR PURPOSE.  See the GNU General Public License
for more details.

You should have received a copy of the GNU General Public License along
with Bash; see the file COPYING.  If not, write to the Free Software
Foundation, 675 Mass Ave, Cambridge, MA 02139, USA.

$PRODUCES builtin.c

$BUILTIN builtin
$FUNCTION builtin_builtin
$SHORT_DOC builtin [shell-builtin [arg ...]]
Run a shell builtin.  This is useful when you wish to rename a
shell builtin to be a function, but need the functionality of the
builtin within the function itself.
$END

#include "../shell.h"
extern char *this_command_name;

/* Run the command mentioned in list directly, without going through the
   normal alias/function/builtin/filename lookup process. */
builtin_builtin (list)
     WORD_LIST *list;
{
  extern Function *find_shell_builtin (), *builtin_address ();
  Function *function;
  register char *command;

  if (!list)
    return (EXECUTION_SUCCESS);

  command = (list->word->word);
#if defined (DISABLED_BUILTINS)
  function = builtin_address (command);
#else /* !DISABLED_BUILTINS */
  function = find_shell_builtin (command);
#endif /* !DISABLED_BUILTINS */

  if (!function)
    {
      builtin_error ("%s: Not a shell builtin!", command);
      return (EXECUTION_FAILURE);
    }
  else
    {
      this_command_name = command;
      list = list->next;
      return ((*function) (list));
    }
}


\include{module}
\include{foreign}
\include{hack}
\include{summary}

\bibliographystyle{name}
\bibliography{abc,esprit,manual}

\documentstyle[manual,twoside,psfig]{report}

\newcommand{\predicate}[3]{\item[#1({\it #3})]\hfil\mbox{}\\}
\newcommand{\prefix}[2]{\item[#1 {\it #2}]\hfil\mbox{}\\}
\newcommand{\infix}[3]{\item[{\it #1} #2 {\it #3}]\hfil\mbox{}\\}
\newcommand{\noargpredicate}[1]{\item[#1]\hfil\mbox{}\\}
\newcommand{\function}[4]{\item[{\it #1}] {\bf #2(\it #4})\\}
\newcommand{\noargfunction}[2]{\item[{\it #1}] {\bf #2()}\\}
\newcommand{\macro}[4]{\item[{\it #1}] {\bf #2(\it #4})\\}
\newcommand{\noargmacro}[2]{\item[{\it #1}] {\bf #2}\\}
\newcommand{\variable}[2]{\item[{\it #1}] {\bf #2}\\}
\newcommand{\tty}[1]{\mbox{\verb@#1@}}
\newcommand{\bug}[1]{\footnote{BUG: #1}}
\newcommand{\version}{1.5.0, August 1990}
\newcommand{\pow}[2]{{#1}^{#2}}
\itemsep 0pt

\makeindex
\sloppy
%\psdraft

%\includeonly{intro}

\begin{document}

\begin{titlepage}

\newlength{\uvawidth}
\settowidth{\uvawidth}{\LARGE University of Amsterdam}

\newcommand{\uvaaddress}{%
\parbox[b]{\uvawidth}{%
    \begin{center}
	\LARGE
	      University of Amsterdam \\[3mm]
	\small
	Dept. of Social Science Informatics (SWI) \\
	    Herengracht 196, 1016 BS~~ Amsterdam \\
		    The Netherlands \\
		Tel. (+31) 20 5252073
    \end{center}}}

\mbox{}\vspace{-1in}
\parbox{\textwidth}{%
    \makebox[\textwidth]{%
	\UvA{1in}%
	\hfill%
	\raisebox{-12pt}{\uvaaddress}
	\hfill%
	\SWI{1in}}
}
\vfil\vfil\vfil
\begin{center}
	{\Huge \bf SWI-Prolog 1.5	\\[3mm]
	 \LARGE Reference Manual}	\\[1.5cm]
	{\large \it Jan Wielemaker}	\\[7mm]
	{\large jan@swi.psy.uva.nl}
\end{center}
\vfil
\begin{quote}
SWI-Prolog is a Prolog implementation based on a subset of the WAM
(Warren Abstract Machine, \cite{Warren:83b}). SWI-Prolog has been
designed and implemented such that it can easily be modified for
experiments with logic programming and the relation between logic
programming and other programming paradigms (such as the object
oriented PCE environment, \cite{P1098:C1.6}).  SWI-Prolog has a rich
set of built-in predicates and reasonable performance, which makes it
possible to develop substantial applications in it.  The current
version offers a module system, garbage collection and an interface to
the C language.

This document gives an overview of the features, system limits and
built-in predicates.
\end{quote}
\vfil
\vfil
\begin{quote}
Copyright \copyright\ 1990 Jan Wielemaker
\end{quote}
\end{titlepage}

\pagestyle{esprit}
\newcommand{\bottomleft}{\mbox{}}
\newcommand{\bottomright}{\mbox{}}

{\parskip 0pt
\tableofcontents
}

\include{intro}
This file is builtin.def, from which is created builtin.c.
It implements the builtin "builtin" in Bash.

Copyright (C) 1987, 1989, 1991 Free Software Foundation, Inc.

This file is part of GNU Bash, the Bourne Again SHell.

Bash is free software; you can redistribute it and/or modify it under
the terms of the GNU General Public License as published by the Free
Software Foundation; either version 1, or (at your option) any later
version.

Bash is distributed in the hope that it will be useful, but WITHOUT ANY
WARRANTY; without even the implied warranty of MERCHANTABILITY or
FITNESS FOR A PARTICULAR PURPOSE.  See the GNU General Public License
for more details.

You should have received a copy of the GNU General Public License along
with Bash; see the file COPYING.  If not, write to the Free Software
Foundation, 675 Mass Ave, Cambridge, MA 02139, USA.

$PRODUCES builtin.c

$BUILTIN builtin
$FUNCTION builtin_builtin
$SHORT_DOC builtin [shell-builtin [arg ...]]
Run a shell builtin.  This is useful when you wish to rename a
shell builtin to be a function, but need the functionality of the
builtin within the function itself.
$END

#include "../shell.h"
extern char *this_command_name;

/* Run the command mentioned in list directly, without going through the
   normal alias/function/builtin/filename lookup process. */
builtin_builtin (list)
     WORD_LIST *list;
{
  extern Function *find_shell_builtin (), *builtin_address ();
  Function *function;
  register char *command;

  if (!list)
    return (EXECUTION_SUCCESS);

  command = (list->word->word);
#if defined (DISABLED_BUILTINS)
  function = builtin_address (command);
#else /* !DISABLED_BUILTINS */
  function = find_shell_builtin (command);
#endif /* !DISABLED_BUILTINS */

  if (!function)
    {
      builtin_error ("%s: Not a shell builtin!", command);
      return (EXECUTION_FAILURE);
    }
  else
    {
      this_command_name = command;
      list = list->next;
      return ((*function) (list));
    }
}


\include{module}
\include{foreign}
\include{hack}
\include{summary}

\bibliographystyle{name}
\bibliography{abc,esprit,manual}

\documentstyle[manual,twoside,psfig]{report}

\newcommand{\predicate}[3]{\item[#1({\it #3})]\hfil\mbox{}\\}
\newcommand{\prefix}[2]{\item[#1 {\it #2}]\hfil\mbox{}\\}
\newcommand{\infix}[3]{\item[{\it #1} #2 {\it #3}]\hfil\mbox{}\\}
\newcommand{\noargpredicate}[1]{\item[#1]\hfil\mbox{}\\}
\newcommand{\function}[4]{\item[{\it #1}] {\bf #2(\it #4})\\}
\newcommand{\noargfunction}[2]{\item[{\it #1}] {\bf #2()}\\}
\newcommand{\macro}[4]{\item[{\it #1}] {\bf #2(\it #4})\\}
\newcommand{\noargmacro}[2]{\item[{\it #1}] {\bf #2}\\}
\newcommand{\variable}[2]{\item[{\it #1}] {\bf #2}\\}
\newcommand{\tty}[1]{\mbox{\verb@#1@}}
\newcommand{\bug}[1]{\footnote{BUG: #1}}
\newcommand{\version}{1.5.0, August 1990}
\newcommand{\pow}[2]{{#1}^{#2}}
\itemsep 0pt

\makeindex
\sloppy
%\psdraft

%\includeonly{intro}

\begin{document}

\begin{titlepage}

\newlength{\uvawidth}
\settowidth{\uvawidth}{\LARGE University of Amsterdam}

\newcommand{\uvaaddress}{%
\parbox[b]{\uvawidth}{%
    \begin{center}
	\LARGE
	      University of Amsterdam \\[3mm]
	\small
	Dept. of Social Science Informatics (SWI) \\
	    Herengracht 196, 1016 BS~~ Amsterdam \\
		    The Netherlands \\
		Tel. (+31) 20 5252073
    \end{center}}}

\mbox{}\vspace{-1in}
\parbox{\textwidth}{%
    \makebox[\textwidth]{%
	\UvA{1in}%
	\hfill%
	\raisebox{-12pt}{\uvaaddress}
	\hfill%
	\SWI{1in}}
}
\vfil\vfil\vfil
\begin{center}
	{\Huge \bf SWI-Prolog 1.5	\\[3mm]
	 \LARGE Reference Manual}	\\[1.5cm]
	{\large \it Jan Wielemaker}	\\[7mm]
	{\large jan@swi.psy.uva.nl}
\end{center}
\vfil
\begin{quote}
SWI-Prolog is a Prolog implementation based on a subset of the WAM
(Warren Abstract Machine, \cite{Warren:83b}). SWI-Prolog has been
designed and implemented such that it can easily be modified for
experiments with logic programming and the relation between logic
programming and other programming paradigms (such as the object
oriented PCE environment, \cite{P1098:C1.6}).  SWI-Prolog has a rich
set of built-in predicates and reasonable performance, which makes it
possible to develop substantial applications in it.  The current
version offers a module system, garbage collection and an interface to
the C language.

This document gives an overview of the features, system limits and
built-in predicates.
\end{quote}
\vfil
\vfil
\begin{quote}
Copyright \copyright\ 1990 Jan Wielemaker
\end{quote}
\end{titlepage}

\pagestyle{esprit}
\newcommand{\bottomleft}{\mbox{}}
\newcommand{\bottomright}{\mbox{}}

{\parskip 0pt
\tableofcontents
}

\include{intro}
\include{builtin}
\include{module}
\include{foreign}
\include{hack}
\include{summary}

\bibliographystyle{name}
\bibliography{abc,esprit,manual}

\input{doc.ind}

\end{document}


\end{document}


\end{document}


\end{document}
