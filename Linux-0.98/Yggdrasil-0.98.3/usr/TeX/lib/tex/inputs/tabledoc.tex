% 
% S-Tables Quick Command Summary 
% 
\input verbatim 
\input stables 
\tolerance=10000 
\centerline{\bf S-Tables v1.0} 
\vfil 
\centerline{Robert Nilsson} 
\centerline{Texas A\&M University} 
\vfil 
\begintable 
\multicolumn2\bf\hfil Types of Commands\hfil\eltt 
Start/End|\stpar{2.5in}{\noindent These are the commands 
for starting and ending 
the table}\elt 
Columns Separators|\stpar{2.5in}{\noindent These are 
used to separate the columns 
in the tables}\elt 
Row Separators|\stpar{2.5in}{\noindent These are used 
to separate the rows}\elt 
Configuration|\stpar{2.5in}{\noindent These are used to set up the functioning 
of the tables such as the width 
of a thick rule, whether the internal rules are 
thin or thick, etc.}\elt 
Specials|\stpar{2.5in}{\noindent These include any commands 
that do not fit into 
the above categories}\endtable 
\vfil 
\begintable 
\multicolumn3\bf\hfil Start/End\hfil\eltt 
Command|Arguments|Description\eltt 
"1\begintable1\hfill|none|Start a table\hfill\el 
"1\begintableto1\hfill|width|\stpar{2.5in}{\noindent Start a table with the 
specified width.  The table will be stretched until it is `width' wide}\el 
"1\begintablesp1\hfill|stretch amount|\stpar{2.5in}{\noindent Start 
a table and stretch 
it `stretch amount' wider than it would normally be}\elt 
"1\endtable1\hfill|none|End the table\hfill\endtable 
\vfil 
\begintable 
\multicolumn3\bf\hfil Column Separators\hfil\eltt 
Command|Arguments|Description\eltt 
"1|1\hfill|none|\stpar{2.5in}{\noindent End a column and place a vertical rule 
of the default width between the columns (do not use this at the end of a 
line)}\el 
"1\|1\hfill|none|\stpar{2.5in}{\noindent Same as above but the vertical rule 
will be suppressed}\el 
"1\vt1\hfill|none|\stpar{2.5in}{\noindent Same as above but the vertical rule 
will be thin}\el 
"1\vtt1\hfill|none|\stpar{2.5in}{\noindent Same as above but the vertical rule 
will be thick}\el 
"1\vttt1\hfill|thickness|\stpar{2.5in}{\noindent Same as above 
but the vertical 
rule will be `thickness' wide}\endtable 
\eject 
\begintable 
\multicolumn3\bf\hfil Row Separators\hfil\eltt 
Command|Arguments|Description\eltt 
"1\el1\hfill|none|\stpar{2.5in}{\noindent End a 
line and don't put a rule under 
it.  (Do 
not use this after the last line of the table, use "1\endtable1)}\el 
"1\elt1\hfill|none|\stpar{2.5in}{\noindent Same as 
above except put a thin rule under 
the line}\el 
"1\eltt1\hfill|none|\stpar{2.5in}{\noindent Same as 
above except put a thick rule under 
the line}\el 
"1\elttt1\hfill|thickness|\stpar{2.5in}{\noindent Same 
as above except put a rule of 
width `thickness' under the line}\el 
"1\elspec1\hfill|none|\stpar{2.5in}{\noindent This command is used to set up 
rules under rows that DO NOT span the entire row.  It in effect indicates that 
the next row will specify the rule to be used under the current row.  This is 
especially useful when using with the row spanning commands.  This introduces 
a subclass, the horizontal rule commands}\elt 
\multicolumn3\hfil Horizonal Rule Command Subset\hfil\elt 
"1\trule1\hfill|none|\stpar{2.5in}{\noindent Places a thin horizontal rule 
under a column.  This command  
is only for use in conjunction with the "1\elspec1  
command  (To leave a column blank, i.e. no rule, just leave it blank)}\el 
"1\ttrule1\hfill|none|\stpar{2.5in}{\noindent Same as  
above but the rule will be   
thick}\el 
"1\tttrule1\hfill|thickness|\stpar{2.5in}{\noindent Same as above but the rule 
will be `thickness' thick}\endtable 
\vfil 
You may be wondering what the difference between the "1\elt1 and the "1\trule1 
command is.  The "1\elt1 will end the line and draw a thin rule under it.  The 
"1\trule1 works in conjunction with the "1\elspec1 to generate a special rule. 
The special rule line is entered the same way a regular row will be entered. 
For example, a normal row would look like: 
\vfil 
"1This|is|a|Test\elt1 
\vfil 
\noindent This will produce a row with a thin rule under it.  To produce the 
same effect without a rule under the column `is' the following would be used: 
\vfil 
"1This|is|a|Test\elspec 
\trule||\trule|\trule\el1 
\vfil 
\noindent Notice that the vertical bars are used.  The macro is starting a new 
row and the vertical bars need to be included if you want them to continue 
through the line.  (There is no need to only use the "1|1, any other column 
separator is also valid). 
\eject 
\begintable 
\multicolumn3\bf\hfill Configuration\hfill\eltt 
Variable|Value|Description\eltt 
\multicolumn3\hfill Dimensions\hfill\elt 
"1\stablesthinline1\hfill|dimension|\stpar{2.5in}{\noindent This variable 
contains the width of a thin rule in the table.  The default value is 
0.4pt and it may be changed with the command: 
  
"1\stablesthinline=<dimen>1 
  
\noindent where "1<dimen>1 is the new width.}\el 
"1\stablesthickline1\hfill|dimension|\stpar{2.5in}{\noindent This 
variable contains the width of a thick rule in the table.  The default 
value is 1pt and it may be changed as above.}\elt 
\multicolumn3\hfill Counters\hfill\elt 
"1\stablestyle1\hfill|0|\stpar{2.5in}{\noindent Center the table using 
the current "1\hsize1.  This is the default setting and it may be 
changed by the following command: 
  
"1\stablestyle=1$n$ 
  
\noindent where $n$ is the new value (0, 1, 2, or 3)}\el 
|1|Left justify the table\hfill\el 
|2|Right justify the table\hfill\el 
|3|No justification\hfill\elt 
\multicolumn3\hfill If Statements\hfill\elt 
"1\ifstablesinternalthin1\hfill|true|\stpar{2.5in}{\noindent Make the 
internal rules of the table thin.  This sets the vertical rule 
generated by the "1|1.  To set the value of this variable the following 
command must be used: 
  
"1\stablesinternalthintrue1 
  
\noindent Please note the word `if' is removed and the word `true' has 
been appended to the end.  The value after this command will be true. 
To set it to false append the word `false' instead of `true'.}\el 
|false|\stpar{2.5in}{\noindent Use thick internal rules (where the "1|1 
is used)}\elspec 
|\trule|\trule\el 
"1\ifstablesborderthin1\hfill|true|\stpar{2.5in}{\noindent Use thin rules 
for the border of the table}\el 
|false|\stpar{2.5in}{\noindent Use thick rules for the border of the 
table.  This is the default.}\endtable 
\vfil 
All settings in the configuration section should be used {\bf OUTSIDE} 
the table.  The results of changing a setting inside the table will be 
unpredictable, and undesirable. 
  
There are two more settings that need to be discussed.  First is the 
element buffering.  There are two definitions that are used for this: 
"1\stablesleft1 and "1\stablesright1.  The default settings are as 
follows: 
\vfil 
"1\def\stablesleft{\quad\hfil} 
\def\stablesright{\hfil\quad}1 
\vfil 
\noindent To change these, simply redefine them. 
  
The other setting is the strut.  If you are interested in resetting 
this, the \TeX book should provide sufficient information (The strut 
is used to hold up the box). 
\eject 
\centerline{\bf Specials} 
{\baselineskip=14pt 
\vskip .25in 
This section will be broken into three parts:  the spanning commands, the 
paragraph commands, and miscellaneous information. 
  
First of all we have two (actually three, but I'll discuss the third later) 
spanning commands.  They are "1\multicolumn1 and "1\multirow1.  To use 
"1\multicolumn1 to span several columns the command will be: 
  
"1\multicolumn1$n$ and your data here. 
  
\noindent The $n$ specifies the number of columns to span across.  For 
example, if a table has 3 columns and you want a title across the top, 
$n$ would be 3.  Omit each column separator that is spanned across (in 
this case none would be used).  When this command is used the buffering is 
suspended on the spanning column, so it is necessary to put "1\hfil1's around 
the data in the spanning column to center it. 
  
"1\multirow1 works slightly differently.  The number of rows to span is 
specified in the same way 
as the number of columns in the "1\multicolumn1 macro,  
but the text to be spanned must be placed in curly braces directly after: 
  
"1\multirow1$n$"1{<horizontal material>}1 
  
\noindent The "1<horizontal material>1 will be vertically 
centered in the number  
of spanned rows.  The horizontal rules are not automatically omitted under the 
columns of the rows being spanned.  The "1\elspec1 command 
must be used to omit  
these rules.  There will be an  
example at the end of the documentation of this. 
  
The paragraph commands are "1\stpar1 and "1\stparrow1.  The  
format for "1\stpar1 
is: 
  
"1\stpar{<dimen>}{<vertical material>}1 
  
\noindent The "1<dimen>1 is the width of the paragraph (the "1\hsize1) and the 
"1<vertical material>1 is the paragraph. 
  
"1\stparrow1 will do the same thing as "1\stpar1 but it will also perform the 
function of "1\multirow1.  It is a composite  
command and the only way to span a 
paragraph across multiple rows.  The format is: 
  
"1\stparrow1$n$"1{<dimen>}{<vertical material>}1 
  
\noindent In this command the $n$ is the number of rows to be spanned and the 
other material is the same as in the "1\stpar1 macro.  Please note that the 
rules for spanning multiple rows apply to this macro also (the use of the 
"1\elspec1 command. 
  
To use both multiple rows and multiple columns, specify the "1\multicolumn1 
command first, then the "1\multirow1 or "1\stparrow1. 
  
The last point I would like to make concerns the use of varying width vertical 
rules.  If a thin vertical rule runs into a thick vertical rule there will 
be an offset.  The default for this offset is to the left.  There are two ways 
to change the setting.  The first is by using an `r' after any of the "1\vt1 
commands.  For example "1\vttr1 will produce a thick vertical rule right 
justified on any wider rules.  The other method is by using the 
"1\ifstablesright1 setting.  A true setting will line up all vertical rules 
generated by the "1|1 on the right.  A false setting will make the vertical 
rules generated by the "1|1 left justified (the default). 
  
In all of the specials using a $n$ parameter, if the number to be used is 
greater than 9, it must be placed in curly braces ("1{}1). 
} 
\vfill\eject 
\centerline{\bf Examples} 
\vskip .25in 
This section will give some example tables and the code to generate them 
organized from simple to complex. 
\vskip .25in 
\leftline{\bf Example 1} 
\vskip .125in 
"*\begintable 
Ck\#\vt Date\vt Memo\vt Debit\vt Credit\vt Balance\eltt 
245|8--2|Rent|\$ \hfill 250.00||\$ \hfill 436.29\el 
246|8--2|Danson Electric|\$ \hfill 49.28||\$ \hfill 387.01\el 
247|8--5|Jeff's Grocery|\$ \hfill 35.88||\$ \hfill 351.13\el 
248||Void|||\el 
249|8--10|Danson Times|\$ \hfill 19.00||\$ \hfill 332.13\el 
250|8--14|Pizza Palace|\$ \hfill 9.95||\$ \hfill 322.18\el 
251|8--15|Jones Hardware|\$ \hfill 45.20||\$ \hfill 276.98\el 
252|8--15|Deposit||\$ \hfill 255.81|\$ \hfill 532.79\el 
253|8--21|Account Fee|\$ \hfill .85||\$ \hfill 531.94\el 
254|8--29|Telephone Co.|\$ \hfill 21.19||\$ \hfill 510.75\endtable* 
\vskip .125in 
\begintable 
Ck\#\vt Date\vt Memo\vt Debit\vt Credit\vt Balance\eltt 
245|8--2|Rent|\$ \hfill 250.00||\$ \hfill 436.29\el 
246|8--2|Danson Electric|\$ \hfill 49.28||\$ \hfill 387.01\el 
247|8--5|Jeff's Grocery|\$ \hfill 35.88||\$ \hfill 351.13\el 
248||Void|||\el 
249|8--10|Danson Times|\$ \hfill 19.00||\$ \hfill 332.13\el 
250|8--14|Pizza Palace|\$ \hfill 9.95||\$ \hfill 322.18\el 
251|8--15|Jones Hardware|\$ \hfill 45.20||\$ \hfill 276.98\el 
252|8--15|Deposit||\$ \hfill 255.81|\$ \hfill 532.79\el 
253|8--21|Account Fee|\$ \hfill .85||\$ \hfill 531.94\el 
254|8--29|Telephone Co.|\$ \hfill 21.19||\$ \hfill 510.75\endtable 
\vfill\eject 
\leftline{\bf Example 2} 
\vskip .125in 
"*\begintable 
\multicolumn6 \hfill Account Activity for August\hfill\eltt 
Ck\#\vt Date\vt Memo\vtt Debit\vt Credit\vtt Balance\eltt 
245|8--2|Rent\vtt\$ \hfill 250.00|\vtt\$ \hfill 436.29\el 
246|8--2|Danson Electric\vtt\$ \hfill 49.28|\vtt\$ \hfill 387.01\el 
247|8--5|Jeff's Grocery\vtt\$ \hfill 35.88|\vtt\$ \hfill 351.13\el 
248||Void\vtt|\vtt\el 
249|8--10|Danson Times\vtt\$ \hfill 19.00|\vtt\$ \hfill 332.13\el 
250|8--14|Pizza Palace\vtt\$ \hfill 9.95|\vtt\$ \hfill 322.18\el 
251|8--15|Jones Hardware\vtt\$ \hfill 45.20|\vtt\$ \hfill 276.98\el 
252|8--15|Deposit\vtt|\$ \hfill 255.81\vtt\$ \hfill 532.79\el 
253|8--21|Account Fee\vtt\$ \hfill .85|\vtt\$ \hfill 531.94\el 
254|8--29|Telephone Co.\vtt\$ \hfill 21.19|\vtt\$ \hfill 510.75\endtable* 
\vskip .125in 
\begintable 
\multicolumn6 \hfill Account Activity for August\hfill\eltt 
Ck\#\vt Date\vt Memo\vtt Debit\vt Credit\vtt Balance\eltt 
245|8--2|Rent\vtt\$ \hfill 250.00|\vtt\$ \hfill 436.29\el 
246|8--2|Danson Electric\vtt\$ \hfill 49.28|\vtt\$ \hfill 387.01\el 
247|8--5|Jeff's Grocery\vtt\$ \hfill 35.88|\vtt\$ \hfill 351.13\el 
248||Void\vtt|\vtt\el 
249|8--10|Danson Times\vtt\$ \hfill 19.00|\vtt\$ \hfill 332.13\el 
250|8--14|Pizza Palace\vtt\$ \hfill 9.95|\vtt\$ \hfill 322.18\el 
251|8--15|Jones Hardware\vtt\$ \hfill 45.20|\vtt\$ \hfill 276.98\el 
252|8--15|Deposit\vtt|\$ \hfill 255.81\vtt\$ \hfill 532.79\el 
253|8--21|Account Fee\vtt\$ \hfill .85|\vtt\$ \hfill 531.94\el 
254|8--29|Telephone Co.\vtt\$ \hfill 21.19|\vtt\$ \hfill 510.75\endtable 
\vfill\eject 
\leftline{\bf Example 3} 
\vskip .125in 
"*\begintable 
\multirow2{\#}\vt\multirow2{Date}\vt\multirow2{Memo}\vt Debit/Credit\elspec 
|||\trule\el 
|||Balance\eltt 
\multirow2{245}|\multirow2{8--2}|\multirow2{Rent}|\$ \hfill 250.00\elspec 
|||\trule\el 
|||\$ \hfill 436.29\elttt{.7pt} 
\multirow2{246}|\multicolumn2 \hfill \stparrow2{1.5in}{\TeX can do 
row \& column spanners \& paragraph formatting within them}  \hfill|\$ 
\hfill 49.28 \elspec 
|||\trule\el 
|\multicolumn 2|\$ \hfill 387.01\elttt{.7pt} 
\multirow2{247}|\multirow2{8--5}|\multirow2{Jeff's Grocery}|\$ \hfill 
35.88 \elspec 
|||\trule\el 
|||\$ \hfill 351.13\elttt{.7pt} 
\multirow2{248}||\multirow2{Void}|\elspec 
|||\el 
|||\elttt{.7pt} 
\multirow2{249}|\multirow2{8--10}|\multirow2{Danson Times}|\$ \hfill 
19.00 \elspec 
|||\trule\el 
|||\$ \hfill 332.13\endtable* 
\vskip .125in 
\begintable 
\multirow2{\#}\vt\multirow2{Date}\vt\multirow2{Memo}\vt Debit/Credit\elspec 
|||\trule\el 
|||Balance\eltt 
\multirow2{245}|\multirow2{8--2}|\multirow2{Rent}|\$ \hfill 250.00\elspec 
|||\trule\el 
|||\$ \hfill 436.29\elttt{.7pt} 
\multirow2{246}|\multicolumn 2 \hfill \stparrow2{1.5in}{\TeX can do 
row \& column spanners \& paragraph formatting within them}  
\hfill|\$ \hfill 49.28 \elspec 
|||\trule\el 
|\multicolumn 2|\$ \hfill 387.01\elttt{.7pt} 
\multirow2{247}|\multirow2{8--5}|\multirow2{Jeff's Grocery}|\$ \hfill 
35.88 \elspec 
|||\trule\el 
|||\$ \hfill 351.13\elttt{.7pt} 
\multirow2{248}||\multirow2{Void}|\elspec 
|||\el 
|||\elttt{.7pt} 
\multirow2{249}|\multirow2{8--10}|\multirow2{Danson Times}|\$ \hfill 
19.00 \elspec 
|||\trule\el 
|||\$ \hfill 332.13\endtable 
\vfill\eject 
 
Source for table on page 3: 
 
"*\begintable 
\multicolumn3\bf\hfill Configuration\hfill\eltt 
Variable|Value|Description\eltt 
\multicolumn3\hfill Dimensions\hfill\elt 
"1\stablesthinline1\hfill|dimension|\stpar{2.5in}{\noindent This variable 
contains the width of a thin rule in the table.  The default value is 
0.4pt and it may be changed with the command: 
  
"1\stablesthinline=<dimen>1 
  
\noindent where "1<dimen>1 is the new width.}\el 
"1\stablesthickline1\hfill|dimension|\stpar{2.5in}{\noindent This 
variable contains the width of a thick rule in the table.  The default 
value is 1pt and it may be changed as above.}\elt 
\multicolumn3\hfill Counters\hfill\elt 
"1\stablestyle1\hfill|0|\stpar{2.5in}{\noindent Center the table using 
the current "1\hsize1.  This is the default setting and it may be 
changed by the following command: 
  
"1\stablestyle=1$n$ 
  
\noindent where $n$ is the new value (0, 1, 2, or 3)}\el 
|1|Left justify the table\hfill\el 
|2|Right justify the table\hfill\el 
|3|No justification\hfill\elt 
\multicolumn3\hfill If Statements\hfill\elt 
"1\ifstablesinternalthin1\hfill|true|\stpar{2.5in}{\noindent Make the 
internal rules of the table thin.  This sets the vertical rule 
generated by the "1|1.  To set the value of this variable the following 
command must be used: 
  
"1\stablesinternalthintrue1 
  
\noindent Please note the word `if' is removed and the word `true' has 
been appended to the end.  The value after this command will be true. 
To set it to false append the word `false' instead of `true'.}\el 
|false|\stpar{2.5in}{\noindent Use thick internal rules (where the "1|1 
is used)}\elspec 
|\trule|\trule\el 
"1\ifstablesborderthin1\hfill|true|\stpar{2.5in}{\noindent Use thin rules 
for the border of the table}\el 
|false|\stpar{2.5in}{\noindent Use thick rules for the border of the 
table.  This is the default.}\endtable* 
 
\medskip 
Source for the verbatim.tex used: 
 
"*\chardef\other=12 \newskip\ttglue \ttglue=.5em plus.25em minus.15em 
\def\ttverbatim{\begingroup 
  \catcode`\\=\other   \catcode`\{=\other  \catcode`\}=\other 
  \catcode`\$=\other   \catcode`\&=\other  \catcode`\#=\other 
  \catcode`\%=\other   \catcode`\_=\other  \catcode`\^=\other 
  \catcode`\"=\other     \catcode`\|=\other   \catcode`\~=\other 
  \obeyspaces \obeylines \tt} 
\catcode`\"=\active \def"#1{\ttverbatim \spaceskip\ttglue% \let^^M=\ 
\def\readit##1#1{##1\endgroup}\expandafter\readit}* 
 
\vfil\eject 
 
\leftline{\bf Example 4} 
\vskip .125in 
"*\begintable 
Account|Ck\#|Debit|Credit|Balance\eltt 
\stparrow3{3in}{\noindent The Lyons Investment Memorial Student Fund following 
specifications 11.2.3 of the U.S. Governmental Code CCA1} 
|123|\$\hfill 1,000.00||\$\hfill 20,000\elspec 
|\trule|\trule|\trule|\trule\el 
|124|\$\hfill 200.00||\$\hfill 19,800\elspec 
|\trule|\trule|\trule|\trule\el 
|||\$\hfill 4,000.00|\$\hfill 23,800\elttt{.7pt} 
\multicolumn4\hfil\stpar{5.25in}{At the end of the physical year 1990 the 
balance in the account for Lyons Investment Memorial Student Fund will be 
tallied and the results will be published as  
per Governmental Code 3.4.2 of the 
last payable week in the session.  The value presented here is a projection of 
the actual that will be available.}\hfil|\$\hfill 25,000\endtable* 
\vskip .125in 
\begintable 
Account|Ck\#|Debit|Credit|Balance\eltt 
\stparrow3{3in}{\noindent The Lyons Investment Memorial Student Fund following 
specifications 11.2.3 of the U.S. Governmental Code CCA1} 
|123|\$\hfill 1,000.00||\$\hfill 20,000\elspec 
|\trule|\trule|\trule|\trule\el 
|124|\$\hfill 200.00||\$\hfill 19,800\elspec 
|\trule|\trule|\trule|\trule\el 
|||\$\hfill 4,000.00|\$\hfill 23,800\elttt{.7pt} 
\multicolumn4\hfil\stpar{5.25in}{At the end of the physical year 1990 the 
balance in the account for Lyons Investment Memorial Student Fund will be 
tallied and the results will be published  
as per Governmental Code 3.4.2 of the 
last payable week in the session.  The value presented here is a projection of 
the actual that will be available.}\hfil|\$\hfill 25,000\endtable 
\bye 
