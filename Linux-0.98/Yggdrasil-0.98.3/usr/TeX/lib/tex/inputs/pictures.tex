%%%  File:  pictures.tex
\input mfpic \magnification=\magstep1
\opengraphsfile{pics}\noindent
\picture[12]{0}{10}{0}{10}
\label[cc]{1}{9}{A}\label[cc]{1}{1}{B}
\label[cc]{3}{5}{C}\label[cc]{9}{9}{D}
\label[cc]{9}{1}{E}\label[cc]{7}{5}{F}
\arrow{(1,8.5),(1,1.5)}
\arrow{(1.5,8.5),(2.5,5.5)}
\arrow{(2.5,4.5),(1.5,1.5)}
\arrow{(1.5,9),(8.5,9)}
\arrow{(9,8.5),(9,1.5)}
\arrow{(8.5,8.5),(7.5,5.5)}
\arrow{(7.5,4.5),(8.5,1.5)}
\arrow{(3.5,5),(6.5,5)}
\arrow{(1.5,1),(8.5,1)}
\caption{Commutative diagram example.}
\endpicture

\noindent \picture[20]{-3}{3}{-3}{3} \axes 
\function{-2,2,0.1,((x**3)-x)/3}\endpicture

\noindent \picture[30]{-1}{1}{-1}{1}
\parafcn{0,6,0.1,(cosd(t*150)*cosd(t*90),
                 cosd(t*150)*sind(t*90))}
\cdisk{(0,0),0.25} \endpicture

\noindent \picture[20]{-0.5}{4}{-0.5}{4}
\axes \rect{(0,0),(1,0.5)}
\rectshade{(0,0),(1,0.5)}
\darkershade \rect{(1,0),(2,1)}
\rectshade{(1,0),(2,1)} \darkershade
\rect{(2,0),(3,2)} \rectshade{(2,0),(3,2)}
\caption{A Bar Graph.} \endpicture

\picture[30]{-1}{1.4}{-1}{1.1}
\linedir{(0,0),0,1}
\wedgefill{(0.3,0.2),0,60,1}
\linedir{(0,0),60,1}
\wedgeshade{(0,0),60,105,1}
\linedir{(0,0),105,1}
\arcth{(0,0),60,360,1}
\caption{A Pie Chart.} \endpicture

\closegraphsfile \end

